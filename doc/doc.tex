\documentclass[aspectratio=169]{beamer}

% Pacotes de codificação, idioma e GRÁFICOS
\usepackage[utf8]{inputenc}
\usepackage[T1]{fontenc}
\usepackage[brazil]{babel}
\usepackage{lmodern}
\usepackage{graphicx} % Obrigatório para imagens

% --- CONFIGURAÇÃO DE CORES (VERMELHO E PRETO) ---
\definecolor{MyBlack}{RGB}{10, 10, 10}      
\definecolor{MyDarkGray}{RGB}{30, 30, 30}   
\definecolor{MyRed}{RGB}{220, 20, 60}       
\definecolor{MyWhite}{RGB}{240, 240, 240}   

\setbeamercolor{background canvas}{bg=MyBlack}
\setbeamercolor{normal text}{fg=MyWhite}
\setbeamercolor{frametitle}{fg=MyRed, bg=MyDarkGray}
\setbeamercolor{title}{fg=MyRed}
\setbeamercolor{subtitle}{fg=MyWhite} 
\setbeamercolor{author}{fg=MyWhite}    
\setbeamercolor{item}{fg=MyRed}
\setbeamercolor{subitem}{fg=MyRed}

\setbeamertemplate{navigation symbols}{}

% --- DADOS DA CAPA ---
\title{Forno Elétrico IoT}
\subtitle{Requisitos Técnicos e Proposta de Valor}

% Nomes dos participantes
\author{
	Helton Abadia \\
	Manassés Oiola \\
	Matheus Grossi \\
	Talles Mello
}

\titlegraphic{
	\centering
	% Certifique-se de ter o arquivo logo_fiap.png na pasta
	\includegraphics[width=4cm]{logo_fiap.png}
}

\date{\today}

\begin{document}
	
	% --- SLIDE 1: CAPA ---
	\begin{frame}
		\titlepage
	\end{frame}
	
	% --- SLIDE 2: ARQUITETURA ---
	\begin{frame}{Arquitetura}
		\centering
		% Certifique-se de ter o arquivo mapa.png na pasta
		\includegraphics[width=0.9\textwidth, height=0.8\textheight, keepaspectratio]{mapa.png}
	\end{frame}
	
	% --- SLIDE 3: REQUISITOS (HARDWARE) ---
	\begin{frame}{Requisitos – Hardware e Interface}
		\begin{itemize}
			\item \textbf{Sensoriamento:} \\
			Uso de NTC para medir a temperatura interna do forno.
			
			\vspace{0.6cm} 
			
			\item \textbf{Controle de potência:} \\
			Ponte H driver de potência acionada por PWM para ajuste proporcional da energia aplicada.
			
			\vspace{0.6cm}
			
			\item \textbf{Interface local:} \\
			Display TFT exibindo temperatura, potência e estado (ligado/desligado).
		\end{itemize}
	\end{frame}
	
	% --- SLIDE 4: REQUISITOS (CONECTIVIDADE) ---
	\begin{frame}{Requisitos – Conectividade e Nuvem}
		\begin{itemize}
			\item \textbf{Conectividade:} \\
			Envio de dados via MQTT e HTTP REST para plataformas Ubidots e ThingSpeak. Broker MQTT será utilizado somente para testes ou algum acionamento.
			
			\vspace{0.5cm}
			
			\item \textbf{Dashboards na nuvem:} \\
			Monitoramento remoto de tempo, temperatura e potência, além de comandos de controle.
			
			\vspace{0.5cm}
			
		\end{itemize}
	\end{frame}
	
	% --- SLIDE 5: VALOR AGREGADO (EFICIÊNCIA) ---
	\begin{frame}{Valor Agregado – Monitoramento e Segurança}
		\begin{itemize}
			\item \textbf{Monitoramento remoto em tempo real:} \\
			O cliente acompanha a temperatura, tempo de uso e programado (timer) e estado do forno de qualquer lugar.
			
			\vspace{0.5cm}
			
			\item \textbf{Controle inteligente de potência:} \\
			Maior eficiência energética e redução de custos operacionais.
			
			\vspace{0.5cm}
			
		\end{itemize}
	\end{frame}
	
	% --- SLIDE 6: VALOR AGREGADO (EXPERIÊNCIA) ---
	\begin{frame}{Valor Agregado – Experiência e Mercado}
		\begin{itemize}
			\item \textbf{Interface amigável:} \\
			Display local e dashboards intuitivos facilitam o uso e a tomada de decisão.
			
			\vspace{0.5cm}
			
			\item \textbf{Dados históricos e análises:} \\
			Relatórios de desempenho permitem otimizar processos e prever padrões de uso.
			
			\vspace{0.5cm}
			
			\item \textbf{Inovação e diferenciação:} \\
			Agrega valor ao produto, tornando-o mais competitivo no mercado de eletrodomésticos inteligentes.
		\end{itemize}
	\end{frame}
	
\end{document}